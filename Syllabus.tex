\documentclass[11pt, a4paper]{article}
%\usepackage{geometry}
\usepackage[inner=1.5cm,outer=1.5cm,top=2.5cm,bottom=2.5cm]{geometry}
\pagestyle{empty}
\usepackage{graphicx}
\usepackage{fancyhdr, lastpage, bbding, pmboxdraw}
\usepackage[usenames,dvipsnames]{color}
\definecolor{darkblue}{rgb}{0,0,.6}
\definecolor{darkred}{rgb}{.7,0,0}
\definecolor{darkgreen}{rgb}{0,.6,0}
\definecolor{red}{rgb}{.98,0,0}
\usepackage[colorlinks,pagebackref,pdfusetitle,urlcolor=darkblue,citecolor=darkblue,linkcolor=darkred,bookmarksnumbered,plainpages=false]{hyperref}
\renewcommand{\thefootnote}{\fnsymbol{footnote}}

\pagestyle{fancyplain}
\fancyhf{}
\lhead{ \fancyplain{}{Course Name} }
%\chead{ \fancyplain{}{} }
\rhead{ \fancyplain{}{\today} }
%\rfoot{\fancyplain{}{page \thepage\ of \pageref{LastPage}}}
\fancyfoot[RO, LE] {page \thepage\ of \pageref{LastPage} }
\thispagestyle{plain}

%%%%%%%%%%%% LISTING %%%
\usepackage{listings}
\usepackage{caption}
\DeclareCaptionFont{white}{\color{white}}
\DeclareCaptionFormat{listing}{\colorbox{gray}{\parbox{\textwidth}{#1#2#3}}}
\captionsetup[lstlisting]{format=listing,labelfont=white,textfont=white}
\usepackage{verbatim} % used to display code
\usepackage{fancyvrb}
\usepackage{acronym}
\usepackage{amsthm}
\VerbatimFootnotes % Required, otherwise verbatim does not work in footnotes!



\definecolor{OliveGreen}{cmyk}{0.64,0,0.95,0.40}
\definecolor{CadetBlue}{cmyk}{0.62,0.57,0.23,0}
\definecolor{lightlightgray}{gray}{0.93}



\lstset{
%language=bash,                          % Code langugage
basicstyle=\ttfamily,                   % Code font, Examples: \footnotesize, \ttfamily
keywordstyle=\color{OliveGreen},        % Keywords font ('*' = uppercase)
commentstyle=\color{gray},              % Comments font
numbers=left,                           % Line nums position
numberstyle=\tiny,                      % Line-numbers fonts
stepnumber=1,                           % Step between two line-numbers
numbersep=5pt,                          % How far are line-numbers from code
backgroundcolor=\color{lightlightgray}, % Choose background color
frame=none,                             % A frame around the code
tabsize=2,                              % Default tab size
captionpos=t,                           % Caption-position = bottom
breaklines=true,                        % Automatic line breaking?
breakatwhitespace=false,                % Automatic breaks only at whitespace?
showspaces=false,                       % Dont make spaces visible
showtabs=false,                         % Dont make tabls visible
columns=flexible,                       % Column format
morekeywords={__global__, __device__},  % CUDA specific keywords
}

%%%%%%%%%%%%%%%%%%%%%%%%%%%%%%%%%%%%
\begin{document}
\begin{center}
{\Large \textsc{Introduction to Baking}}
\end{center}
\begin{center}
Spring 2019
\end{center}
%\date{September 26, 2018}

\begin{center}
\rule{6in}{0.4pt}
\begin{minipage}[t]{.75\textwidth}
\begin{tabular}{llcccll}
\textbf{Instructor:} & Ningyi Zheng && &  & \textbf{Time:} & F 14:00 -- 17:00 \\
\textbf{Email:} &  \href{ningyi.Zheng@berkeley.edu}{ningyi.Zheng@berkeley.edu} & & & & \textbf{Place:} & 4105 Dwinelle Hall
\end{tabular}
\end{minipage}
\rule{6in}{0.4pt}
\end{center}
\vspace{.5cm}
\setlength{\unitlength}{1in}
\renewcommand{\arraystretch}{2}

\vskip.15in
\noindent\textbf{Office Hours:} After class, or by appointment, or post your questions on piazza

\vskip.15in
\noindent\textbf{Main References:} %\footnotemark
This is a list of various interesting and useful books that will be touched during the course. Those books are not required but recommended.
\begin{itemize}

\item Gisslen, Wayne. {\textit{Professional Baking}}, Hoboken, 6th edition. 2000.
\item Figona, Paula. {\textit{How Baking Works}}, Cambridge University Press, 2005.

\end{itemize} 

% \footnotetext{Downloadable ebook versions are available on AeLP.}

\vskip.15in
\noindent\textbf{Objectives:}  
\begin{itemize}
\item Identify and explain baking terms, ingredients, equipment and tools.\\
\item Employ safe food handling practices using contemporary guidelines.
\end{itemize} 
\vskip.15in
\noindent\textbf{Skills:}
\begin{itemize}
\item {Scale and measure ingredients.}
\item {Prepare yeast dough, quick breads, pies, cookies, cakes, icing, pate choux, and savory baking.}
\item {Produce baked products using commercial ingredients and equipment.}
\end{itemize} 


\vspace*{.15in}

\noindent \textbf{Course Outline:}
\begin{center} 
\begin{minipage}{5in}
\begin{flushleft}
%Chapter 1 \dotfill ~$\approx$ 3 days \\
{\color{darkgreen}{\Rectangle}} This course has been developed to provide first year culinary students with knowledge and skills necessary to produce quality baked goods. This is a course designed to teach basic baking skills.
\end{flushleft}
\end{minipage}
\end{center}

\vspace*{.15in}
\noindent\textbf{Grading Policy:} Attendances (30\%),  Group Project 1 (20\%), Group Project 2 (20\%), Final project (30\%). %Four Projects (40\% = 4 * 10\%)

\vskip.15in
\noindent\textbf{Important Dates:}
\begin{center} \begin{minipage}{3.8in}
\begin{flushleft}
Project \#1      \dotfill February 16, 2019 \\
Project \#2      \dotfill March 15, 2019  \\
%Project Deadline \dotfill ~Month Day \\
Final project      \dotfill May 6, 2019  \\
\end{flushleft}
\end{minipage}
\end{center}

\vskip.15in
\noindent\textbf{Course Policy:}  
\begin{itemize}
\item Please arrive on time. I will do row-call before every class 

\end{itemize}

\vskip.15in
\noindent\textbf{Class Policy:}  
\begin{itemize}
\item Regular attendance is essential and expected.
\end{itemize}

\vskip.15in
\noindent\textbf{Academic Honesty:}   Lack of knowledge of the academic honesty policy is not a reasonable explanation for a violation.

\newpage

% Course Outline
\textbf {\large Tentative Course Outline}:

\begin{table}[h!]
\normalsize % The size of the table text can be changed depending on content. Remove if desired.
\begin{tabular}{ | c | c | }
\hline
\textbf{Week} & \textbf{Content} \\
\hline
Week 1 & \begin{minipage}{.85\textwidth}
\begin{itemize} \itemsep-0.4em
	\vspace{1mm}
	\item Introduction/Course outline
\item Safety/Sanitation
\item Scaling/Measuring
\item Equipment/Tools
	\vspace{1mm}
\end{itemize}
\end{minipage} \\
\hline
Week 2 & \begin{minipage}{.85\textwidth}
\begin{itemize} \itemsep-0.4em
	\vspace{1mm}
	\item Learn how chocolate is made
	\item Profile chocolate in tasting, smelling, appearance, etc.
	\item How to work with chocolate and learn basic garnishes
	\vspace{1mm}
\end{itemize}
\end{minipage} \\
\hline
Week 3 & \begin{minipage}{.85\textwidth}
\begin{itemize} \itemsep-0.4em
	\vspace{1mm}
	\item Learn  Muffin, Creaming methods
	\item Concentration this week on Muffin/Creaming Methods
	\vspace{1mm}
\end{itemize}
\end{minipage} \\
\hline
Week 4 & \begin{minipage}{.85\textwidth}
\begin{itemize} \itemsep-0.4em
	\vspace{1mm}
	\item Learn how Biscuit Method works for different applications.
	\item Understanding of different types of pie/sweet tart dough and uses
	\vspace{1mm}
\end{itemize}
\end{minipage} \\
\hline
Week 5 & \begin{minipage}{.85\textwidth}
\begin{itemize} \itemsep-0.4em
	\vspace{1mm}
	\item Stirred vs. Baked Custard and the differences
	\item Learn the uses for these basic products in applications of other pastry and desserts
		\item \textbf {First project due}
	\vspace{1mm}
\end{itemize}
\end{minipage} \\
\hline
Week 6 & \begin{minipage}{.85\textwidth}
\begin{itemize} \itemsep-0.4em
	\vspace{1mm}
	\item Difference between dough in taste for savory vs. sweet
	\item Handling, rolling, shaping of dough
	\item Using the right dough for the right filling
    \item Finishes for pies
	\vspace{1mm}
\end{itemize}
\end{minipage} \\
\hline
Week 7 & \begin{minipage}{.85\textwidth}
\begin{itemize} \itemsep-0.4em
	\vspace{1mm}
	\item \textbf {Spring week}
	\vspace{1mm}
\end{itemize}
\end{minipage} \\
\hline
Week 8 & \begin{minipage}{.85\textwidth}
\begin{itemize} \itemsep-0.4em
	\vspace{1mm}
	\item Types of cookies
    \item Mixing methods
    \item Storing different cookie dough

	\vspace{1mm}
\end{itemize}
\end{minipage} \\
\hline

Week 9 & \begin{minipage}{.85\textwidth}
\begin{itemize} \itemsep-0.4em
	\vspace{1mm}
	\item We will tour Whole Foods at 6th
	\item Look and discuss all the departments and new items never seen before
	\vspace{1mm}
\end{itemize}
\end{minipage} \\
\hline

Week 10 & \begin{minipage}{.85\textwidth}
\begin{itemize} \itemsep-0.4em
	\vspace{1mm}
	\item Yeast-Fresh vs. Dried, how does it work
    \item Procedures for mixing yeast dough
    \item Storing breads/dough for future uses
    \item \textbf {Second Project Due}
	\vspace{1mm}
\end{itemize}
\end{minipage} \\
\hline
Week 11 & \begin{minipage}{.85\textwidth}
\begin{itemize} \itemsep-0.4em
	\vspace{1mm}
	\item Learning the differences between Puff/Croissant/Danish dough
    \item What is the best applications/uses for these item
	\vspace{1mm}
\end{itemize}
\end{minipage} \\
\hline
Week 12 & \begin{minipage}{.85\textwidth}
\begin{itemize} \itemsep-0.4em
	\vspace{1mm}
	\item Pate a Choux, Puff Pastry, Fillo
    \item Learn the different applications for each
    \item Talk about Meringues-American, Swiss, Italian
	\vspace{1mm}
\end{itemize}
\end{minipage} \\
\hline

Week 13 & \begin{minipage}{.85\textwidth}
\begin{itemize} \itemsep-0.4em
	\vspace{1mm}
	\item \textbf {RRR week}
	\item Review different types of ingredients used for savory baking
    \item Look at trends for baking ideas in the industry.
	\vspace{1mm}
\end{itemize}
\end{minipage} \\
\hline
Week 14 & \begin{minipage}{.85\textwidth}
\begin{itemize} \itemsep-0.4em
	\vspace{1mm}
	\item Final Week!
	\item \textbf{Final project Due} 
	\vspace{1mm}
\end{itemize}
\end{minipage} \\
\hline
\end{tabular} 
\end{table}



%%%%%% THE END 
\end{document} 



